%\title{LaTeX Portrait Poster Template}
%%%%%%%%%%%%%%%%%%%%%%%%%%%%%%%%%%%%%%%%%
% a0poster Portrait Poster
% LaTeX Template
% Version 1.0 (22/06/13)
%
% The a0poster class was created by:
% Gerlinde Kettl and Matthias Weiser (tex@kettl.de)
% 
% This template has been downloaded from:
% http://www.LaTeXTemplates.com
%
% License:
% CC BY-NC-SA 3.0 (http://creativecommons.org/licenses/by-nc-sa/3.0/)
%
%%%%%%%%%%%%%%%%%%%%%%%%%%%%%%%%%%%%%%%%%

%----------------------------------------------------------------------------------------
% PACKAGES AND OTHER DOCUMENT CONFIGURATIONS
%----------------------------------------------------------------------------------------

\documentclass[a0,portrait]{a0poster}
\usepackage[utf8]{inputenc}
\usepackage{multicol} % This is so we can have multiple columns of text side-by-side
\columnsep=100pt % This is the amount of white space between the columns in the poster
\columnseprule=3pt % This is the thickness of the black line between the columns in the poster

\usepackage[svgnames]{xcolor} % Specify colors by their 'svgnames', for a full list of all colors available see here: http://www.latextemplates.com/svgnames-colors

\definecolor{opticlimb}{rgb}{0,0.65,0.65}

%\usepackage{times} % Use the times font
%\usepackage{palatino} % Uncomment to use the Palatino font
%\usepackage{helvet}
\usepackage{avant}

\usepackage{graphicx} % Required for including images
\graphicspath{{figures/}} % Location of the graphics files
\usepackage{booktabs} % Top and bottom rules for table
\usepackage[font=small,labelfont=bf]{caption} % Required for specifying captions to tables and figures
\usepackage{amsfonts, amsmath, amsthm, amssymb,bm} % For math fonts, symbols and environments
\usepackage{wrapfig} % Allows wrapping text around tables and figures

\renewcommand{\familydefault}{\sfdefault}

\newcommand{\E}[1]{\mathbb{E}[#1]}
\newcommand{\parametre}[1]{\color{OrangeRed} #1 \color{Black}}
\newcommand{\etat}[1]{\color{ForestGreen} #1 \color{Black}}
\newcommand{\control}[1]{\color{Blue} #1 \color{Black}}
%\newcommand{\donnee}[1]{\color{BurntOrange} #1 \color{Black}}

\begin{document}

%----------------------------------------------------------------------------------------
% POSTER HEADER 
%----------------------------------------------------------------------------------------

% The header is divided into two boxes:
% The first is 75% wide and houses the title, subtitle, names, university/organization and contact information
% The second is 25% wide and houses a logo for your university/organization or a photo of you
% The widths of these boxes can be easily edited to accommodate your content as you see fit

\begin{minipage}[b]{0.75\linewidth}
\veryHuge \color{Navy} \textbf{Non linear Mixed Effects Models:\\
Bridging the gap between Independent Metropolis Hastings and Variational Inference} \color{Black}\\[2cm] % Title
%\Huge\textit{Identification de système dynamique}\\[2cm] % Subtitle
\huge \textbf{Belhal Karimi$^{1,2}$, M. Lavielle$^{1,2}$, E. Moulines$^{2}$}\\[0.5cm] % Author(s)
\huge INRIA$^1$, CMAP École Polytechnique$^2$\\[0.4cm] % University/organization
\Large \texttt{belhal.karimi@cmap.polytechnique.edu}\\
\end{minipage}
%
\begin{minipage}[b]{0.25\linewidth}

\begin{center}
%\includegraphics[width=12cm]{logo.jpg}\\
%\includegraphics[width=15cm]{opticlimb.png}\\
\end{center}
\end{minipage}

\vspace{1cm} % A bit of extra whitespace between the header and poster content

%----------------------------------------------------------------------------------------

\begin{multicols}{2} % This is how many columns your poster will be broken into, a portrait poster is generally split into 2 columns

%----------------------------------------------------------------------------------------
% ABSTRACT
%----------------------------------------------------------------------------------------

%\color{Navy} % Navy color for the abstract

%\begin{abstract}
%
%Sed fringilla tempus hendrerit. Vestibulum ante ipsum primis in faucibus orci luctus et ultrices posuere cubilia Curae; Etiam ut elit sit amet metus lobortis consequat sit amet in libero. Lorem ipsum dolor sit amet, consectetur adipiscing elit. Phasellus vel sem magna. Nunc at convallis urna. isus ante. Pellentesque condimentum dui. Etiam sagittis purus non tellus tempor volutpat. Donec et dui non massa tristique adipiscing. Quisque vestibulum eros eu. Phasellus imperdiet, tortor vitae congue bibendum, felis enim sagittis lorem, et volutpat ante orci sagittis mi. Morbi rutrum laoreet semper. Morbi accumsan enim nec tortor consectetur non commodo nisi sollicitudin. Proin sollicitudin. Pellentesque eget orci eros. Fusce ultricies, tellus et pellentesque fringilla, ante massa luctus libero, quis tristique purus urna nec nibh.
%
%\end{abstract}

%----------------------------------------------------------------------------------------
% INTRODUCTION
%----------------------------------------------------------------------------------------

%\color{SaddleBrown} % SaddleBrown color for the introduction
%\color{Navy} % Navy color
%\color{opticlimb}
\color{DarkSlateGray}

\section{Problem statement}
\color{Navy} % Navy color
%\color{DarkSlateGray} % DarkSlateGray color for the rest of the content
L'avion est aujourd'hui le transport le plus polluant par passager et par kilomètre parcouru. De plus, la quantité de carburant consommée par vol reste un paramètre économique crucial pour les compagnies aériennes. 

Par ailleurs, alors que la phase de montée est la plus grande consommatrice de carburant, celle-ci se fait souvent à vitesse constante et poussée maximale. Parallèlement, une quantité croissante de données sont acquises par les boites noires au cours des vols et celles-ci sont de plus en plus accessibles.
\newline

Dans ce contexte, nous nous sommes intéressés à \textbf{l'utilisation de données de vols pour optimiser des trajectoires de montée dans le but de réduire la consommation de carburant et les émissions de $CO_2$ d'avions de ligne.}

%----------------------------------------------------------------------------------------
% OBJECTIVES
%----------------------------------------------------------------------------------------

\color{DarkSlateGray} % DarkSlateGray color for the rest of the content
%\vspace{1cm}
\section{Model and notations}

Notre finalité est de proposer une méthode permettant de fournir des trajectoires optimales, réalisables et acceptables. Ceci peut se traduire mathématiquement par un \textbf{problème de commande optimale} de la forme suivante :
\begin{equation}
\min_{(\bm{x,u}) \in \mathbb{X} \times \mathbb{U}} J(\bm{x,u})  \qquad       \textrm{tel que} \qquad \frac{d \bm{x} }{dt}=g(\bm{x,u})
\end{equation}
où $\bm{x}$ et $\bm{u}$ désignent les variables d'état et de contrôle de l'aéronef et $g$ est une fonction continue définissant la dynamique du système. La fonction coût $J$ à minimiser est ici supposée différentiable et définie à partir de la consommation de carburant et de la distance parcourue.


Le caractère \textbf{réalisable} des solutions de tel problème est assuré par le respect de la contrainte dynamique $\frac{d \bm{x} }{dt}=g(\bm{x,u})$. Celle-ci correspond à un modèle du comportement de l'avion, qui doit être le plus proche possible de la réalité, de sorte que la solution du problème réduise effectivement la masse de carburant consommée. Une possibilité simple de modèle (réf.\cite{hull}) est illustré par la figure \ref{fig-dyn} et décrite ci-dessous:

\begin{align} \label{eq-sys}
\arraycolsep=1.4pt\def\arraystretch{1.8}
\left\{ 
\begin{array}{ll}
\dot{h} &=  V \sin \gamma \\
\dot{V} &=  \frac{T \cos \alpha - D - mg \sin \gamma}{m}\\
\dot{\gamma} &= \frac{T \sin \alpha + L- mg \cos \gamma}{mV}\\
\dot{m} &= -C_{sp} T
\end{array} \right.
\end{align}
Le vecteur des variables d'état $\bm{x}$ est ici composé de l'altitude $h$ , la vitesse $V$, la pente $\gamma$ et la masse de l'avion $m$, tandis que les variables de contrôle $\bm{u}$ sont l'angle d'attaque $\alpha$ et le régime moteur $N_1$.
\newline

Les autres éléments intervenants dans les équations précédentes, à savoir $T,D,L$ et $C_{sp}$, désignent respectivement la force de poussée totale, la force de traînée, la force de portance et la consommation spécifique moyenne des moteurs. On suppose ici que ceux-ci peuvent être décrits comme des fonctions des variables d'état et de contrôle. \textbf{Notre objectif est donc, pour un avion donné, d'estimer ces fonctions à partir d'enregistrements de vol afin d'avoir une contrainte dynamique décrivant le plus fidèlement possible le comportement et les performance de celui-ci.}

%----------------------------------------------------------------------------------------
% MATERIALS AND METHODS
%----------------------------------------------------------------------------------------
%\vspace{1cm}
\section{Convergence Theorems}

Des modèles paramétriques des fonctions $T,D,L$ et $C_{sp}$ ont été définis. En isolant ces éléments inconnus dans les membres de droite, les trois dernières équations du système \eqref{eq-sys} permettent de définir le \textbf{problème de régression} suivant:

\begin{equation} \label{eq-regpb}
\left\{
\arraycolsep=1.4pt\def\arraystretch{1.5}
\begin{array}{ll}
Y_x &= T \cos(\alpha) - D + \varepsilon_x \\
Y_z &= T \sin(\alpha) + L + \varepsilon_z \\
Y_c &= C_{sp} T+ \varepsilon_c,
\end{array}
\right.
\end{equation}
où $\{\varepsilon_k\}_{k=1}^3$ sont les erreurs de notre modèle.

La principale difficulté de ce problème réside dans la troisième équation, qui est non-linéaire et ne permet pas d'identifier séparément $C_{sp}$ et $T$.
\newline

Pour tenter de résoudre ce problème d'apprentissage statistique et faire face à cette difficulté, trois approches différentes ont été testées :
\begin{description}
\item[Méthode 1 -] Ne pas identifier $C_{sp}$, en le remplaçant par un modèle prédéfini, et réaliser les trois régressions séparément ;
\item[Méthode 2 -] Utiliser les deux premières équations simultanément pour identifier $T$, $D$ et $L$, puis injecter l'estimation de $T$ dans la dernière équation afin d'estimer $C_{sp}$ ;
\item[Méthode 3 -] Se servir d'une approche de type Newton (réf. \cite{bonnans06}) pour réaliser toutes les trois régressions simultanément.
\end{description}
\vspace{1cm}

Une des motivations des deux dernières approches citées ci-dessus est le résultats connu en statistiques selon lequel \textbf{la précision peut être améliorée en résolvant des problèmes de régression simultanément plutôt que séparément} (réf. \cite{stein61}).

La mesure de précision utilisée ici a été l'\textbf{Erreur de Prédiction Espérée} :
\begin{equation}
\mbox{EPE}_k = \E{|Y_k - \hat{Y}_k|^2}, \qquad k\in \{x,z,c\}
\end{equation}
où $\hat{Y}_k$ désigne la prédiction de $Y_k$, un des membres de gauche des équations \eqref{eq-regpb}.

%----------------------------------------------------------------------------------------
% RESULTS 
%----------------------------------------------------------------------------------------
%\vspace{1cm}
\section{Incremental in Practice}

Résultats obtenus à partir d'une base de 1000 variables enregistrées toutes les secondes au cours de 78 vols d'un unique avion.

\subsection{Mixed effects models}
Résultats obtenus à partir d'une base de 1000 variables enregistrées toutes les secondes au cours de 78 vols d'un unique avion.

\subsection{PK-PD datasets}
Résultats obtenus à partir d'une base de 1000 variables enregistrées toutes les secondes au cours de 78 vols d'un unique avion.

%\begin{center}\vspace{1cm}
%\includegraphics[width=0.8\linewidth]{resultats2}
%\captionof{figure}{Comparaison de la consommation de carburant avec et sans les commande optimisées }
%\end{center}\vspace{1cm}

%----------------------------------------------------------------------------------------
% CONCLUSIONS
%----------------------------------------------------------------------------------------

%\color{SaddleBrown} % SaddleBrown color for the conclusions to make them stand out
%\color{Navy}
%\vspace{1cm}
\section{Conclusion}
\color{Navy} % Navy color

Les résultats précédents permettent d'illustrer les conclusions suivantes:
\begin{itemize}
\item Les meilleures méthodes d'identification semblent être les méthodes 1 et 3, cette dernière étant bien plus complexe à mettre en œuvre.
\item Les modèles identifiés à partir de la méthode 1 nous permettent déjà d'obtenir des résultats acceptables.
\end{itemize}
\textbf{Ceci illustre bien que l'accès à de grandes quantités de données rend parfois des approches simples compétitives face à des solutions plus sophistiquées.}
\vspace{1cm}

Par la suite, des modèles et méthodes de résolution différentes seront étudiés et nos expérimentations se feront sur des échantillons 10 fois plus importants.

Il est également important de préciser que cette solution est déjà utilisée dans le produit \textbf{OptiClimb, commercialisé par la société Safety Line. Il permet d'obtenir 10\% d'économies de carburant sur la phase de montée.}

\color{DarkSlateGray} % Set the color back to DarkSlateGray for the rest of the content

%----------------------------------------------------------------------------------------
% FORTHCOMING RESEARCH
%----------------------------------------------------------------------------------------

%\section{Forthcoming Research}


 %----------------------------------------------------------------------------------------
% REFERENCES
%----------------------------------------------------------------------------------------

%\subsection*{Références}
\small
\nocite{*} % Print all references regardless of whether they were cited in the poster or not
%\bibliographystyle{plain} % Plain referencing style
\bibliographystyle{siam}
\bibliography{edf} % Use the example bibliography file sample.bib

%----------------------------------------------------------------------------------------
% ACKNOWLEDGEMENTS
%----------------------------------------------------------------------------------------

%\section*{Acknowledgements}


%----------------------------------------------------------------------------------------

\end{multicols}
\end{document}